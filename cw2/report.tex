\documentclass[fleqn]{article}

\usepackage{cmap}
\usepackage[T2A]{fontenc}
\usepackage[utf8]{inputenc}
\usepackage[english,russian]{babel}

\usepackage[14pt]{extsizes}
\usepackage[left=20mm, top=15mm, right=15mm, bottom=15mm, nohead, footskip=10mm]{geometry}

\usepackage{listings}
\usepackage{verbatim}
\usepackage{titlesec}
\usepackage{graphicx}
\usepackage{color}
\usepackage[colorlinks=true,linkcolor=black,anchorcolor=black,citecolor=black,filecolor=black,menucolor=black,runcolor=black,urlcolor=black]{hyperref}

\usepackage{courier}

\usepackage{amsmath,amsfonts,amssymb,amsthm,mathtools}

\begin{document}
\begin{titlepage}
\newpage
\begin{center}
Министерство цифрового развития, связи и массовых коммуникаций Российской Федерации\\
ФГБОУ высшего образования \\
"Сибирский Государственный Университет Телекоммуникаций и Информатики" (СибГУТИ)

Кафедра прикладной математики и кибернетики
\end{center}
\vspace{9em}
\begin{center}
Курсовая работа \\
по дисциплине <<Объектно-ориентированное программирование>>
\end{center}

\begin{center}
Тема: Игра <<Жизнь>>
\end{center}

\vspace{15em}

\begin{center}
\hfillВыполнил: студент 2 курса группы ИП-012 \\
\hfillМаланов Роман Игоревич \\
\hfillПроверил: доцент кафедры ПМиК \\ 
\hfillСитняковская Е.И.
\end{center}

\vspace{3em}

\vfill

\begin{center}
Новосибирск, 2021
\end{center}
\end{titlepage}

\definecolor{gray}{rgb}{0.3, 0.4, 0.4}

\lstset{
  language=C++,
  numbers=left,                   % where to put the line-numbers
  stepnumber=1,                   % the step between two line-numbers.
  numbersep=15pt,                  % how far the line-numbers are from the code
  backgroundcolor=\color{white},  % choose the background color. You must add \usepackage{color}
  showspaces=false,               % show spaces adding particular underscores
  tabsize=2,                      % sets default tabsize to 2 spaces
  breaklines=true,                % sets automatic line breaking
  breakatwhitespace=true,         % sets if automatic breaks should only happen at whitespace
  extendedchars=\true,
  inputencoding=utf8,
  basicstyle=\sffamily\footnotesize,
  commentstyle=\fontseries{lc}\selectfont\itshape\color{gray},
  keywordstyle=\color{blue},
  stringstyle=\ttfamily\color{red},
}

\tableofcontents

\newpage

\section{Введение}

\subsection{Задача:}

Написать программу для игры в "Жизнь".

\subsection{Описание игры:}



\section{Описание используемых алгоритмов и функций}

Программа написана на языке Си под платформу Linux с использованием ncurses, 
библиотеки для управления вводом-выводом в терминал.

Программа создаёт в терминале 16 клеток (фишек), заполненных цифрами от 1 до 15
и одной пустой. Управление фишками происходит с помощью мыши.
Нажатие на фишку сдвигает её, если рядом с ней есть пустая клетка.

\begin{itemize}
		\item В функции \verb|main.c| происходит создание всех блоков,
				настройка параметров и отслеживание нажатий клафиш и мыши.

		\item Текущий порядок фишек хранится в массиве \verb+numbers+.

		\item Для хранения фишек, их координат и содержащихся в них чисел
				используется структура \verb|struct _cell|, определённая как тип
				\verb+Cell+.

		\item Функция \verb+randomize()+ перемешивает числа в массиве
				\verb+numbers+. Для перемешивания фишек используется
				клавиша \verb+`r`+, перемешивание происходит до тех пор,
				пока не получаем разрешимую кобинацию (ровно половина
				возможных комбинаций неразрешима). Проверку на разрешимость \
				текущего порядка фишек осуществляет функция \verb+is_solvable()+
				(проверяет чётность текущего порядка фишек, см. 

				\href{https://ru.wikipedia.org/wiki/%D0%98%D0%B3%D1%80%D0%B0_%D0%B2_15#%D0%9C%D0%B0%D1%82%D0%B5%D0%BC%D0%B0%D1%82%D0%B8%D1%87%D0%B5%D1%81%D0%BA%D0%BE%D0%B5_%D0%BE%D0%BF%D0%B8%D1%81%D0%B0%D0%BD%D0%B8%D0%B5}{Пятнашки:Математическое описание}).

		\item Функции \verb|get_coord()| и \verb|get_cell_idx()| взаимно обратны
				и используются для получения координат фишки по индексам
				и для получения индекса по известным координатам соответственно.

		\item Функция \verb|get_motion()| проверяет, есть ли рядом с данной
				фишкой свободная клетка. Возвращает тип \verb|Motion|, 
				направление, в котором можно сместить фишку.

		\item Функция \verb|make_movement| смещает фишку на пустую клетку
				(меняет местами числа в массиве \verb|numbers|).

		\item Функция \verb|check_win()| проверяет, получил ли игрок
				правильное положение фишек (проверяет возрастание чисел
				в массиве \verb|numbers| и положение пустой клетки).
				Проверка происходит только когда фишка с цифрой '1' 
				находится в левом верхнем углу, а пустая клетка в правом нижнем.

\end{itemize}

\newpage

\section{Код программы}

\subsection{Листинг:}

language: C++
\lstinputlisting[title=Creature.hpp]{Creature.hpp}
\lstinputlisting[title=Plant.hpp]{Plant.hpp}
\lstinputlisting[title=LivingCreature.hpp]{LivingCreature.hpp}
\lstinputlisting[title=Prey.hpp]{Prey.hpp}
\lstinputlisting[title=Hunter.hpp]{Hunter.hpp}
\lstinputlisting[title=Game.hpp]{Game.hpp}

Source-файлы:
\lstinputlisting[title=Creature.cpp]{Creature.cpp}
\lstinputlisting[title=Plant.cpp]{Plant.cpp}
\lstinputlisting[title=LivingCreature.cpp]{LivingCreature.cpp}
\lstinputlisting[title=Prey.cpp]{Prey.cpp}
\lstinputlisting[title=Hunter.cpp]{Hunter.cpp}
\lstinputlisting[title=Game.cpp]{Game.cpp}

\subsection{Скриншоты с примером работы:}

%\begin{figure}[h!]
%		\includegraphics[width=0.5\textwidth]{screenshot.jpg}
%		\caption{Состояние при запуске}
%\end{figure}
%\begin{figure}[h!]
%		\includegraphics[width=0.5\textwidth]{screenshot2.jpg}
%		\caption{Перемешивание фишек}
%\end{figure}
%\begin{figure}[h!]
%\includegraphics[width=0.5\textwidth]{screenshot3.jpg}
%		\caption{Собранная головоломка}
%\end{figure}

\end{document}
