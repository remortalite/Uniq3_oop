\documentclass[fleqn]{article}

\usepackage{cmap}
\usepackage[T2A]{fontenc}
\usepackage[utf8]{inputenc}
\usepackage[english,russian]{babel}

\usepackage[12pt]{extsizes}
\usepackage[left=20mm, top=15mm, right=15mm, bottom=15mm, nohead, footskip=10mm]{geometry}

\usepackage{listings}
\usepackage{verbatim}
\usepackage{titlesec}
\usepackage{graphicx}
\usepackage{color}
\usepackage[colorlinks=true,linkcolor=black,anchorcolor=black,citecolor=black,filecolor=black,menucolor=black,runcolor=black,urlcolor=black]{hyperref}

\usepackage{courier}

\usepackage{amsmath,amsfonts,amssymb,amsthm,mathtools}

\begin{document}
\begin{titlepage}
\newpage
\begin{center}
Министерство цифрового развития, связи и массовых коммуникаций Российской Федерации\\
ФГБОУ высшего образования \\
"Сибирский Государственный Университет Телекоммуникаций и Информатики" (СибГУТИ)

Кафедра прикладной математики и кибернетики
\end{center}
\vspace{9em}
\begin{center}
Курсовая работа \\
по дисциплине <<Объектно-ориентированное программирование>>
\end{center}

\begin{center}
Тема: Игра <<Жизнь>>
\end{center}

\vspace{15em}

\begin{center}
\hfillВыполнил: студент 2 курса группы ИП-012 \\
\hfillМаланов Роман Игоревич \\
\hfillПроверил: доцент кафедры ПМиК \\ 
\hfillСитняковская Е.И.
\end{center}

\vspace{3em}

\vfill

\begin{center}
Новосибирск, 2021
\end{center}
\end{titlepage}

\definecolor{gray}{rgb}{0.3, 0.4, 0.4}

\lstset{
  language=C++,
  numbers=left,                   % where to put the line-numbers
  stepnumber=1,                   % the step between two line-numbers.
  numbersep=15pt,                  % how far the line-numbers are from the code
  backgroundcolor=\color{white},  % choose the background color. You must add \usepackage{color}
  showspaces=false,               % show spaces adding particular underscores
  tabsize=2,                      % sets default tabsize to 2 spaces
  breaklines=true,                % sets automatic line breaking
  breakatwhitespace=true,         % sets if automatic breaks should only happen at whitespace
  extendedchars=\true,
  inputencoding=utf8,
  basicstyle=\sffamily\footnotesize,
  commentstyle=\fontseries{lc}\selectfont\itshape\color{gray},
  keywordstyle=\color{blue},
  stringstyle=\ttfamily\color{red},
}

\tableofcontents

\newpage

\section{Введение}

\subsection{Постановка задачи:}

Игроку представляется поле размером $N$ клеток на $N$ клеток (в данной работе используется $N=150$), реализованное в графическом режиме.
Каждую секунду, или по нажатию клавиши \verb|E| сменяется жизненный цикл.
На клетках находятся живые организмы трёх типов:
\begin{itemize}
	\item Тип \textbf{Растение} (\verb+Plant+): является пищей для травоядного, не двигается. Определённое количество растений появляется раз в несколько жизненных циклов.

	\item Тип \textbf{Травоядное животное} (\verb|Prey|): поглощает растения и является пищей для хищника. Каждое травоядное животное каждый жизненный цикл может сдвинуться на одну из соседних клеток, с некоторой вероятностью оставляя после себя травоядное животное. Наступая на клетку с растением, поглощает его. Размножается с вероятностью 35\%.

	\item Тип {\bf Хищник} (\verb"Hunter"): поглощает травоядных, но не взаимодействует с растениями. Хищник каждый ход может сдвинуться на одну из соседних клеток (при голоде может сдвинуться на две клетки). Поглощает всех  травоядных животных, которых встретит. С растениями не взаимодействует. Размножается только после поглощения минимум двух травоядных животных с вероятностью 25\%.
\end{itemize}

Травоядные и хищники обладают системой голода. Изначальный показатель голода для хищников 36, для травоядных 11 (значения подобраны в ходе реализации для достижения умеренно стабильного состояния поля в ходе жизненного цикла).
Каждое перемещение животного отнимает 2 единицы от показателя голода.
Рождение отнимает 4 единицы.
Поглощение прибавляет к показателю голода 4.
Животное погибает, когда показатель голода становится равным нулю.

Растения живут 6 циклов, затем погибают.
Определённое количество растений появляется на поле каждые 1-6 циклов.

\section{Технологии ООП}

Для реализации программы были использованы принципы ООП, такие как: наследование, полиморфизм, инкапсуляция, абстрактный класс.

\section{Реализация}

Программа написана на языке \verb"С++" под платформу Linux с ипользованием графической библиотеки SFML.

Описание классов:
\begin{itemize}
		\item \verb"class Creature":

			Абстрактный класс, класс-родитель для всех классов живых существ.
			В нём задаются координаты объекта и цвет.
			Также оъявляются общие для всех объектов виртуальные функции, например \verb"isDead()" или \verb"step()" (функция, меняющая значения аттрибутов класса каждый жизненный цикл).

		\item \verb"class Plant":

			Класс объекта Растение. Наследуется от класса \verb"Creature". В классе дополнительно приватно определена переменная \verb"m_lifetime" и соответствующая функция \verb"getLifetime()". Графически обозначен зелёным цветом.

		\item \verb"class LivingCreature":

			Класс, наследующийся от класса \verb"Creature", но также являющийся классом-родителем для классов с системой голода (классы \verb"Prey" и \verb"Hunter").
			В классе определены соответствующие переменные для контроля 
			показателя голода и константные переменные для величины 
			изменения показателя голода при определённом действии. 

		\item \verb"class Prey" и \verb"class Hunter":

			Классы объектов Травоядное животное и Хищник. 
			Наследуются от класса \verb"LivingCreature".
			Наследуют аттрибуты родительских классов и переопределяют виртуальные функции.
			Объекты класса Травоядное животное обозначены жёлтым цветом, объекты класса Хищник -- Красным.

		\item \verb"class Game":

			Самый крупный класс в программе.
			Создаёт игровое поле, контролирует весь игровой процесс,отображение фигур, создание и перемещение объектов.

\end{itemize}

\section{Результаты работы}

В ходе работы была реализована программа для игры Жизнь. Параметры программы настраиваемые, при экспериментировании получаются интересные различные постоянно изменяющиеся состояния игрового поля.

\newpage

\section{Скриншоты}

\begin{figure}[h!]
		\includegraphics[width=0.5\textwidth]{screen1}
		\caption{Одно из состояний программы (хищники почти отсутствуют)}
\end{figure}

\begin{figure}[h!]
		\includegraphics[width=0.5\textwidth]{screen2}
		\caption{Одно из состояний программы}
\end{figure}

\newpage

\section{Код программы}

language: C++
\lstinputlisting[title=Creature.hpp]{Creature.hpp}
\lstinputlisting[title=Plant.hpp]{Plant.hpp}
\lstinputlisting[title=LivingCreature.hpp]{LivingCreature.hpp}
\lstinputlisting[title=Prey.hpp]{Prey.hpp}
\lstinputlisting[title=Hunter.hpp]{Hunter.hpp}
\lstinputlisting[title=Game.hpp]{Game.hpp}

Source-файлы:
\lstinputlisting[title=Creature.cpp]{Creature.cpp}
\lstinputlisting[title=Plant.cpp]{Plant.cpp}
\lstinputlisting[title=LivingCreature.cpp]{LivingCreature.cpp}
\lstinputlisting[title=Prey.cpp]{Prey.cpp}
\lstinputlisting[title=Hunter.cpp]{Hunter.cpp}
\lstinputlisting[title=Game.cpp]{Game.cpp}

\end{document}
